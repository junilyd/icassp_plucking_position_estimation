% ICASSP 2019 - Paper on plucking position on the fretted string.
% --------------------------------------------------------------------------
\documentclass{article}
\usepackage{spconf,amsmath,graphicx,amsfonts}
\usepackage{svg}
\usepackage{cite}
\usepackage{tabularx,booktabs,colortbl}
\usepackage{hyperref}
\input macros.tex 
% ------
\title{Estimation of Guitar String, Fret and Plucking Position Using Parametric Pitch Estimation}
% ---------------
\name{Jacob~M\o ller Hjerrild and~Mads~Gr\ae sb\o ll~Christensen}%\thanks{Thanks to XYZ agency for funding.}}
\address{Audio Analysis Lab, CREATE, Aalborg University, Denmark\\
   \texttt{\{jmhh, mgc\}@create.aau.dk}}
%
\begin{document}
\ninept
\maketitle
%
%
% ---- ABSTRACT
%
%
\begin{abstract}
In this paper a method is proposed for the extraction of instrumental controls during guitar performance. Specifically, the activated string and fret as well as the location of the plucking event along the guitar string are extracted from guitar signal recordings. The method is based on a parametric pitch estimator derived from a physically meaningful model that includes inharmonicity. A maximum a posteriori classifier is used for the string and fret classification, which requires training data captured from only one fret. The classifier is tested on recordings of electric and acoustic guitar for every possible training fret and shows accurate results: the average absolute error of string and fret classification is $1.5\%$, while the error rate varies dependent on the fret used for training. The plucking position estimator is the minimizer of the log spectral distance between the amplitudes of the observed signal and the plucking model and it is evaluated as a proof-of-concept, where a clear trend is shown for estimating string, fret and plucking position combinations. The method can reliably extract the control parameters and has a simple parametric framework with physically meaningful parameters.

\end{abstract}
%
\begin{keywords}
 Physical Modeling, Statistical Signal Processing, Machine Learning, Parametric Pitch Estimation, Music Information Retrieval.\vspace{-.8mm}
\end{keywords}
%
%
% ---- INTRODUCTION
%
%
% %%%%%%%%%%%%%%%%%%%%%%%% NOTES  %%%%%%%%%%%%%%%%%
%
% why is the topic of interest ?
%
% - guitar playing style. Where does Jimi Hendrix put his fingers ?
% - Tablature
% - (in general: music transription)
% - guitar tuning
% - parametric pitch estimation
% - physical modeling.
%
% What is the background on potential solutions
%
% What was attempted in the present effort research project.
% 
% ... the simplicity of both the model, the feature set and the complexity of the algorithm a model based approach leads to an effective solution for the proposed.  
% 
% %%%%%%%%%%%%%%%%%%%%%%%%%%%%%%%%%%%%%%%%%%%%%%%
%
\section{Introduction} % (fold)
\label{sec:introduction}
\vspace{-.6mm}
%
Several papers have studied the analysis and synthesis of plucked string instruments e.g., acoustic guitar~\cite{Karjalainen93towardshigh-quality,laurson2001methods} and electric guitars~\cite{sullivan1990extending}. In the present study, we focus on the analysis of guitar signals. The motivation for this study is to understand the factors that influence the sound of well-known guitarists, in order to be able to replicate their sound by extracting the relevant parameters from their recordings. 
To date, there are few papers on extracting information from electric guitar recordings, such as classifying the types of effects used~\cite{abesser2012feature} and estimating the decay time of electric guitar tones~\cite{pate2014predicting}. Other research involved extracting information from related string instruments, such as extracting plucking styles and dynamics for classical guitar~\cite{erkut2000extraction} and electric bass guitar~\cite{abesser:automatic_string_detection_ml}. Recent papers introduce techniques to model the physical interactions of the player with the guitar to synthesize a more realistic guitar sound, such as modeling the interactions of the guitar pick~\cite{germain2009synthesis,evangelista2010player} or fingers~\cite{poirot_nonlinear_interactions_with_string} with the string, and the fingers with the fretboard~\cite{bilbao2015numerical}.

It is well-known that the plucking position and pickup position produce a comb-filtering effect in the spectrum of the guitar signal~\cite{fletcher:plucked_strings,fletcher:principles_of_vibration_and_sound} and that stringed instruments are not perfectly harmonic which some of the first theoretical studies on inharmonicity show~\cite{donkin:acoustics,rayleigh:sound}. Shankland and Coltman~\cite{coltShank} showed how inharmonicity is mainly caused by stiffness and deflection, which was elaborated in~\cite{rossing:science_of_string_instruments}. The well-known piano model of inharmonicity was derived by H. Fletcher in~\cite{fletcher:piano_model} and has been used recently for string classification purposes; the inharmonicity coefficient has been proposed for electric guitar string classification contained in a 48-dimensional feature set~\cite{abesser:automatic_string_detection_ml}, where the inharmonicity coefficient was selected as one of the most discriminative features. A string and fret classification algorithm was proposed~\cite{dittmar:realtime_string_detection}, using a 10 dimensional feature set and an SVM classifier. Large feature sets are prone to overfitting and rarely contribute to simple and meaningful findings in terms of physical cause and effect relationship. However, Barbancho et al.~\cite{barbancho:inharmonicity_tablature} proposed an inharmonicity and amplitude based method for automatic generation of guitar tablature, where they locate the fundamental frequency using heuristic peak finding in the spectrum. Papers that studied the estimation of the plucking event location of the guitar string have used both frequency-domain~\cite{DBLP:conf/icassp/MohamadDH17,traube:pluckin_point_dafx,traube2003extraction} and time-domain~\cite{penttinen2004time} approaches, but only for open strings or by assuming a known pitch or string and fret position. \\
%
\indent In this paper we consider the fretted string scenario, hence the estimation of plucking event location and classification of string and fret. The feature set consists of three physically meaningful parameters which are extracted with a non-linear least squares (NLS) pitch estimator~\cite{nielsen2017fast,multipitch,hansen2018parametric,DBLP:journals/sigpro/ChristensenSJJ08} extended to include inharmonicity.  
We use the inharmonic pitch estimates for training a Bayes maximum a posteriori (MAP) classifier of string and fret combinations which requires training data captured from only one fret on each string, such that a guitar player will be able to swiftly train his guitar on the fly. 
Finally, the proposed method estimates the plucking position for the fretted string scenario, as opposed to~\cite{traube:pluckin_point_dafx,DBLP:conf/icassp/MohamadDH17}, which was done for open strings only. The method is tested on electric and acoustic guitars, recorded for this purpose. % and is available online along with MATLAB code. 
The purpose is to extract the location of interactions of both hands of the guitar player when both hands can be arbitrarily located along a string.
\begin{figure}[h!]\
  \centering
  \centerline{\includegraphics[width=.99\columnwidth]{img/fender_drawing7.png}}\vspace{-2mm}
  \caption{Right hand controls plucking position and left hand controls pitch using the fretboard. One pitch is produced in various positions.
  }\label{fig:guitar_overview}\vspace{-2mm}
\end{figure}
Generally, the left hand changes the pitch and the right hand activates the string vibration by plucking as shown in Fig.~\ref{fig:guitar_overview}. In the following we describe the model of the guitar string. 
% \indent The paper is structured as follows: Section~\ref{sec:signal_model} derives the inharmonic signal model for the guitar string including an ideal model of string excitation. Section~\ref{sec:proposed_method} explains a method to estimate the plucking position on the fretted string after classification of the string/fret combination based on the inharmonic pitch estimates. In Section~\ref{sec:experiments}, we evaluate the method on two data sets: (1) we evaluate the accuracy of the classifier when various frets are used for training, and (2) we evaluate the plucking
% position estimates for fretted strings given various fret and string classes. Finally, Sec.~\ref{sec:conclusion} gives the conclusion. \vspace{-.8mm}
% %
%
%
% ---- STRING AND SIGNAL MODEL
%
%
\vspace{-1.9mm}
\section{String Model} % (fold)
\label{sec:signal_model}
\vspace{-.6mm}
We start by modeling string displacement activated by plucking, before the signal parameters of interest is described. 
The vibrating part of the string has length $L$ and is fixed at $l=0$ and $l=L$ with pinned boundaries. %, hence $y(0,t)=0$ and $y(L,t)=0$. 
For a small displacement $y$, the motion is described by the partial differential equation $\frac{\partial^2 y}{\partial t^2} = c^2\frac{\partial^2 y}{\partial l^2}$, where $c$ is the speed of the transverse wave. The well-known ideal string solution is ~\cite{fletcher:principles_of_vibration_and_sound}
\begin{align}
     y(l,t) = \sum_m \big(A_m \sin{\omega_mt} + A'_m \cos{\omega_mt}\big) \sin{\kappa_ml},
\end{align}
%
where $\omega$ is frequency, $\kappa_m=\omega_m/c$ is the wave number and the amplitude of the $m$th mode is $C_m\!\!=\!\!\sqrt{A_m^2\!+\!{A'}_m^2}$. 
%
The string is modeled with an initial deflection $\delta$ excited at plucking position $P$, by the plucking hand with an edge sharp pick at the $P$th fraction of its length ($0\!\! <\!\! PL\!\! <\!\! L$). There is no initial velocity i.e. $ \frac{\partial y}{\partial l} \!\!= \!\dot{y}(l,0)\!=0, \  \forall l$ and we assume an initial triangular string shape, i.e. 
\begin{equation}
     y(l,0) =\begin{cases}
               \frac{\delta}{P}\frac{l}{L}, \quad & 0\leq l \leq PL\\
               \frac{\delta}{1\!-\!P}(1\!-\!\frac{l}{L}),        \quad & PL\leq l \leq L. 
            \end{cases}\label{eq:string_initialization}
\end{equation}
For a fixed $P$, the m$th$ Fourier coefficients of this string is 
\begin{align}
    C_m(P) \!\! &= \!\! \frac{2}{L}\bigg[\! \frac{\delta}{PL} \! \int_0^{PL}\!\!\!\!\!\!\!\!\! l \sin{\!\frac{m\pi l\!}{L}}\:\! dl\! 
        + \!\frac{\delta}{1\!-\!P}\!\!\! \int_{PL}^L \!\!\!(1\!-\!\frac{l}{L}) \sin{\! \frac{m\pi l\!}{L}}\:\! dl \bigg]\nonumber \\
        &= \frac{2\delta}{m^2 \pi^2 P(1-P)} \sin{m \pi P }, \label{eq:pluck_closed_form}
\end{align}
which explains how timbre changes as a function of plucking position. From~\eqref{eq:pluck_closed_form} it is clear that the harmonic amplitudes decay with $m^{-2}$ with a sinusiodal spectral envelope caused by $P$, independent of pitch. From an open string length $L_{\textup{open}}$ (from bridge to nut) and a given fret $\fret_1$, the corresponding vibrating string length $L_1$ is given by $L_1 = L_{\textup{open}}  2^\frac{-\fret_1}{12}$ (see Fig.~\ref{fig:guitar_overview}).

For an electric or semi-acoustic guitar, the displacement $y(l,t)$ is measured with an electrical transducer (a pickup), which we assume is close to the vibrating string in a fixed location $(l=\lambda)$. For a discrete time sampled signal at time instance $n$ we define the signal
\begin{equation}
     x(n)  \vert_{l=\lambda} \propto y(\lambda, t),
\end{equation}   
where $x(n)$ is the guitar signal recorded with the pickup at $\lambda$. We propose to parametrize $x(n)$ with an inharmonic signal model as explained in the following. 
%
 At time instance $n$, the observed complex-valued signal vector $\vecx \in \mathbb{C}^N$ is represented as $\vecx = [x(0) \, x(1) \, \cdots \, x(N-1)]^T$, with $T$ denoting the transpose. %Do we need this?
A complex signal can ease both notation and computational complexity and a real-valued signal is converted to complex by using the Hilbert transform~\cite{LawrenceMarple1999}. The $n$th entry of $\vecx$ is modeled as an inharmonic sinusoidal part and a noise part i.e.,  
\begin{equation}\label{eq:sigmod1}
  x(n)\! =  \!\sum\limits_{m=1}^{M}\!\! \alpha_{m} \exp\big({j\psi_m(\omega_0,B) n}\big)+v(n), 
  %x(n)\! = \!s(n)\!+\!v(n)\!= \!\sum\limits_{m=1}^{M} \alpha_{m} e^{j(m\omega_0 \sqrt{1+B m^2}) n}+v(n), 
\end{equation}
where $\omega_0$ is the fundamental frequency, $M$ is the number of partials, $\alpha_{m}\!\! =\! A_me^{\phi_m}$ is the complex amplitude of the $m$th partial, $\phi_m$ its phase, $v(n)$ is noise and the instantaneous frequency  $\psi_m(\omega_0,B)$ is derived in~\cite{fletcher:piano_model} as
\begin{equation}
  \psi_m(\omega_0,B) = m \omega_0 \sqrt{1+B m^2}. 
\end{equation}
For ease of notation, we denote it as $\psi_m$ although it is a function of $\omega_0$ and $B$. The model order $M$ can be estimated~\cite{nielsen2017fast,multipitch}, while for the string model, initialized by the triangular shape in~\eqref{eq:string_initialization} we assume a high $M$ at the onset event. In vector-matrix notation the observed signal is modeled as
\begin{equation}\label{eq:xZa}
  \vecx = \matZ \vecalpha + \vecv,
\end{equation} 
where the complex sinusoidal matrix $\matZ \in \mathbb{C}^{N\times M}$ is given by
\begin{align}
  \matZ =& \left[ \vecz(\psi_1) \: \vecz(\psi_2) \: \cdots \: \vecz(\psi_M)\right], \\
  \vecz(\psi_m) =& \left[ 1 \: e^{j\psi_m} \: e^{j\psi_m2} \: \cdots \: e^{j\psi_m(N-1)} \right]^{T},
\end{align}
$\vecalpha = [\alpha_1 \: \cdots \: \alpha_M]^T$ is a vector containing complex amplitudes and $\vecv = [v(0) \: v(1) \: v(N-1)]^T$ contains all noise terms. We denote the unknown and deterministic parameters with $\vectheta$, i.e.
\begin{equation}\label{eq:theta_parameters}
  \vectheta = \{\omega_0, B, \vecalpha\}.
\end{equation}
The amplitudes $\vecalpha$ can be estimated with the least squares, while the other parameters $\omega_0$ and $B$ are non-linear. 
The inharmonic pitch and inharmonicity coefficient estimates $\{\widehat\omega_0, \widehat B \} $ are sufficient for classification of string and fret~\cite{barbancho:inharmonicity_tablature,michelson2018_aes} and the estimated amplitude vector $\vecalphahat$ is used for estimation of the plucking position $\widehat{P}$.% on the vibrating string with length $\hat{L}$, implicitly given by the inverse proportionality of the pitch estimate $\omega_0$ and the full string length.
%     
%
%
% ---- PROPOSED METHOD
%
%
%
\section{Proposed Method} % (fold)
\label{sec:proposed_method}
\vspace{-.6mm}
Fig.~\ref{fig:overview} gives an overview of the proposed method. The proposed method is initialized with a detection of the onset event from which one segment is extracted and the following estimation is done on such a segment alone. The feature set in~\eqref{eq:theta_parameters} is extracted as maximum likelihood with the NLS inharmonic pitch estimation method. $\{\widehat\omega_0, \widehat B \} $ are applied to a MAP classifier of string and fret. At the estimated inharmonic frequencies $\widehat{\boldsymbol{\psi}}$, the complex amplitudes $\vecalphahat$ are estimated using least squares. These are used for estimation of plucking position $\widehat{P}$. All details are derived in the following.
In this study the onset detection is considered a solved problem which can be obtained with a filter bank method~\cite{olivier:mirtoolbox_dafx}.% command $\texttt{mironsets(x,'filter','diff','contrast',0.2)}$.
%
%
% ---- PITCH ESTIMATION (Proposed method)
%
%
\subsection{Inharmonic Pitch Estimation} % (fold)
\label{sec:proposed_estimator}
\begin{figure}[t]\
  \centering
  \centerline{\includegraphics[width=.75\columnwidth]{img/block2.png}}\vspace{-2mm}
  \caption{Overview of the proposed method.}\label{fig:overview}\vspace{-2mm}
\end{figure}
The pitch and inharmonicity parameters are estimated by maximizing the likelihood function %probability of the observed data $\vecx$ given the parameters:
\begin{equation}
    \thetahat = \argmax{\vectheta}{\Like(\vectheta | \vecx)} = \argmax{\vectheta}{p(\vecx ; \vectheta)}.
\end{equation}
The observed signal distribution is modeled in circular complex white Gaussian noise with covariance matrix $\C$, i.e., 
\begin{equation} 
    p(\vecx ; \vectheta) = \frac{1}{\pi^N\det{(\C)}} e^{\left(-\vecv^H\C^{-1}\vecv\right)},
\end{equation}
where $\C=\sigma^2\I$ is a diagonal matrix, scaled by an unknown variance $\sigma^2$, where $\I$ is the $N \times N$ identity. By the use of~\eqref{eq:xZa} with $\vecv = \vecx - \matZ \vecalpha$, 
the log-likelihood function is expressed as
\begin{equation}\label{eq:log_likelihood2}
    \ln\Like(\vectheta|\vecx) = -N \ln(\pi) - N \ln \frac{1}{N}|| \vecx-\Z \vecalpha ||_2^2 - N.
\end{equation}
By neglecting all terms that do not dependent on $\omega_0$ and $B$, the maximum likelihood solution is the minimizer of the 2-norm error between the observed signal and the signal model, expressed as
\begin{equation}
  \thetahat =  \argmax{\vectheta}{\Like(\vectheta | \vecx)}  =\argmin{\vectheta}{\norm{\vecx-\Z \vecalpha}_2^2}.
\end{equation}
By substituting $\vecalpha$ with its least squares estimate
\begin{equation}\label{eq:alphahat}
  \vecalphahat = (\Z^H\Z)^{-1}\Z^H\vecx,
\end{equation}
the inharmonic non-linear least squares (NLS) pitch estimator is
\begin{equation}\label{eq:psi_B_est}
  \{ \widehat\omega_0, \widehat B \}= \argmin{\psi_0, B}{|| \vecx - \Z(\Z^H\Z)^{-1}\Z^H\vecx}||_2^2.
\end{equation}
Asymptotically $N(\Z^H\Z)^{-1}\!\!=\!\I|_{N\to\infty}$ , a computational efficient approach is realized as
\begin{equation} \label{eq:ANLS}
    \{\widehat\omega_0, \widehat B \} = \argmax{\psi_0,B}{\vecx^H \matZ \matZ^H \vecx} = \argmax{\psi_0,B}{ \norm{\matZ^H\vecx}_2^2},
\end{equation}
which can be implemented with one DFT. Since $B\!<\!<\!1$, an initial pitch estimate is obtained with $B=0$, and from that we define a narrow two dimensional search grid for the inharmonic pitch ($\omega_0$ and $B$) to ease computational complexity. An optimal grid can be selected using~\cite{jkn:grid_size}. Finally, the amplitudes are estimated using~\eqref{eq:alphahat}.
%
%
%
% ---- CLASSIFICATION (Proposed method)
%
%
%
\subsection{String and Fret Classification} % (fold)
\label{sec:proposed_classification_of_string_and_fret}
The purpose is to classify $\vecx$ as a string and fret position, given the feature vector $\vecphi = [\widehat\omega_0, \widehat B ]^T$ obtained from~\eqref{eq:ANLS}. We have a set of $K$ mutually exclusive classes $\symvec{\Gamma}=\{\gamma_1,\dots,\gamma_K\}$ representing all possible string and fret positions. The MAP-optimal classifier with decision function $\hat{\gamma}(\cdot)\!:\! \mathbb{R}^I \!\!\!\rightarrow \!\!\boldsymbol{\Gamma}  $ is~\cite{mspr}
\begin{align}
    \hat\gamma_{\textup{MAP}}(\vecphi) &= \argmax{\gamma\in\Gamma}{p(\gamma|\vecphi)} = \argmax{\gamma\in\Gamma}{p(\vecphi|\gamma)P(\gamma)}.
\end{align}
We model $\vecphi$ as coming from a normal object with class $\gamma_k$, then the $k$th conditional probability density is   
\begin{equation}
    p(\!\gamma_k\lvert\vecphi)\! = \!\! {{(\!2\pi\!)^{\!\!\frac{\!-\!I}{2}}\!\!\det{\!(\!\boldsymbol{\Lambda}_k\!)\!^{\!\frac{\!-\!1}{2}}\!\!}}}\, \exp\!\!\bigg(\!\!\frac{-(\!\vecphi-\boldsymbol{\mu}_k\!)^T\!\boldsymbol{\Lambda}_k^{-1} (\!\vecphi-\boldsymbol{\mu}_k\!) }{2}\!\! \bigg),
\end{equation}
where the expectation vector $\boldsymbol{\mu}_k$ and covariance matrix $\boldsymbol{\Lambda}_k$ are given from training.
%
%
the covariance matrix is class independent and isotropic, we have that $\boldsymbol{\Lambda}_k=\sigma^2\mathbf{I}$ and it gives the following classification scheme:
\begin{equation}\label{eq:classifier}
  \hat{\gamma}(\vecphi)\!\!=\!{\gamma}_i \;\; \textup{with} \;\; 
  i\!=\!\argmax{k=1,\dots,K}
  {\bigg\{\!
   2\ln\! P(\!\gamma_k)\! -\! \frac{\norm{\vecphi-\!\boldsymbol{\mu}_k}^2}{\sigma^2}\! \bigg\}}\!,
\end{equation}
when neglecting every term that does not depend on $k$ and taking the logarithm.
The classifier in~\eqref{eq:classifier} is the minimizer of the Euclidean distance between the observation and its expectation, with a correction factor of $2\sigma^2 P(\gamma_k)$. The prior $P(\gamma_k)$ can be specified from the number of training samples from class $\gamma_k$. For Euclidean distance, we normalize both features to have an absolute maximum of 1.
%
%
%
% ---- TRAINING (Proposed method)
%
%
%
\subsubsection{Training Procedure} % (fold)
\label{ssub:fast_training}
The purpose of training is to build a model containing the full set of parameters $\{\boldsymbol{\mu}_k, \boldsymbol{\Lambda}_k\}, \forall k$, given a labelled set of $I$ i.i.d. samples $\vecPhi=\{ \vecphi^{(1)},\dots, \vecphi^{(I)} \}$. % where $\vecphi^{(i)}$ is the $i$th estimate $\{\widehat\psi_0^{(i)}, \widehat B^{(i)} \}$ from~\eqref{eq:ANLS}. 
For lowering complexity and increasing the usability for the guitar player, a method is desired that quickly learns the parameters in~\eqref{eq:ANLS}, thus implicitly the inharmonicity related to the guitar playing style. Given the parameters $B_s( \fret_1 )$ and $\omega_{0,s}( \fret_1 )$ of the $s$th vibrating string for a given fret $\fret_1$, the corresponding parameter can computed for any other fret $\fret_2$, using the inharmonicity model derived in~\cite{barbancho:inharmonicity_tablature}
\begin{equation}\label{eq:B_definition}
    \widehat{B}_s( \fret_2 ) =  \widehat{B}_s( \fret_1 ) \: 2^{ \frac{  \fret_2 - \fret_1 }{ 6 }} = \frac{ \pi^3 E_s d_s^4 }{ 64 T_s L^2_{s}(\fret_2) },
\end{equation}
and for the pitch estimates we have that
\begin{equation}\label{eq:omega0_fast}
  \widehat{\omega}_{0,s}( \fret_2 ) =  \widehat{\omega}_{0,s}( \fret_1 ) 2^{\frac{\fret_2-\fret_1}{12}}.
\end{equation}
Hence, the classifier only need to be trained using audio captured from one fret in order to model the parameters of the remaining frets. In~\eqref{eq:B_definition} $E_s$ is Elastic Modulus, $d_s$ is core diameter, $T_s$, which can be considered constants~\cite{rossing:science_of_string_instruments}. %Aside from lowering the computational complexity,~\eqref{eq:B_definition} and~\eqref{eq:omega0_fast} are used in a $13$-fold hold out scheme where one model is trained on data one fret and classify the remaining data 12 frets. 
\vspace{-.8mm}
\subsection{Plucking Position Estimation} % (fold)
\label{sec:proposed_estimation_of_pluck_amplitude}
\vspace{-.6mm}
As shown in Figure~\ref{fig:overview}, the estimation of plucking position $\widehat{P}$ can be estimated after the amplitude estimates of~\eqref{eq:alphahat} have been obtained. $\widehat{P}$ is found by minimizing the log spectral (LS) distance between the observation $\vecalphahat$ and the model $\mathbf{C}$, which is defined as
\begin{equation}
    d_{\textup{LS}}(\vecalphahat,\mathbf{C}(P)) = \sqrt{ \frac{1}{M} \sum_m 10\log_{10} \frac{\lvert\widehat{\alpha}_m\rvert^2}{\lvert C_m(P)\rvert^2} },   
\end{equation} 
where $\vecalphahat = [\widehat{\alpha}_1,\widehat{\alpha}_2,\dots,\widehat{\alpha}_M]^T$ is obtained from~\eqref{eq:alphahat} and $\mathbf{C}(P) = \newline[C_1(P),C_2(P),\dots,C_M(P)]^T$ is the model in~\eqref{eq:pluck_closed_form}. By assuming that the string is ideal the optimization problem is
\begin{equation}
    \widehat{P} = \argmin{P}{\big(d_{\textup{LS}}(\vecalphahat,\mathbf{C}(P))\big)},
\end{equation}
where we emphasize that the model $\mathbf{C}$ can be even more physical meaningful by combining it with a model of the pickup location, such as the model in~\cite{DBLP:conf/icassp/MohamadDH17}.\vspace{-2mm}
%
%
% ---- EVALUATION 
%
%
\section{Evaluation} % (fold)
\label{sec:experiments}
\vspace{-.6mm}
Some tests have been done to evaluate the proposed plucking position estimator along with the string and fret classifier. We remark that we do not compare to~\cite{barbancho:inharmonicity_tablature,michelson2018_aes},  since the classification uses a continuous estimate of both pitch and inharmonicity estimated with the NLS pitch estimator. As opposed to~\cite{barbancho:inharmonicity_tablature,michelson2018_aes} we test on short 40 ms frames. The NLS estimator have been shown to reach the Cramér-Rao lower bound~\cite{nielsen2017fast}, hence we do not test the pitch estimator accuracy. The proposed method was evaluated on recorded data (44.1 kHz). The data and MATLAB code can be downloaded from \url{https://tinyurl.com/yc76blld}. %VBN: https://tinyurl.com/y7ov86ye}. 
The data consists of guitar recordings of the electric and acoustic guitar, each labelled with their respective string and fret combination, namely electric Les Paul Firebrand with Elixir Nanoweb (.10-.54) strings and an acoustic Martin DR with SP (.12-.52) strings. The recorded data of each guitar represents 10 line signal recordings of every string and fret combination from the $0$th to $12$th, hence it consists of 10 recordings of 13 frets and 6 strings which resembles 720 recordings.
For experiments, every recording is segmented from the detected onset time event in to a 40 ms segment. \vspace{-.6mm}
%
%
%
%
%%%%%%%%%%%%%%%%%%%%%%%%%%%%%%%%%%%%%%%%%%%%%%%%%%%%%%%%%%%%%%%%%%%%%%%%%%%%%%%
%
% TEST SETUP TABLE
%
% \begin{table}\centering
% \caption{\label{tbl:experiments}Test setup.}
% \begin{tabularx}{0.4\textwidth}{@{}l*{7}{c}c@{}}
% \toprule
% % General Test Setup  &    \\
% % \midrule
% Sampling rate ($f_s$)  	& 44.1 kHz  \\
% Signal segment duration       	& 40 ms.  \\
% $\textbf{Guitars:}$				& \\
% Martin DRS2 &  with SP	(.12--.52) strings		\\
% Les Paul Firebrand		& with Elixir (.10--.54) strings\\
% \bottomrule
% \end{tabularx}
% \end{table}\vspace{-4mm}
%    
% \subsection{Pitch and Inharmonicity Estimation} % (fold)
% \label{sec:clustermodel_selection}
% Insert graph of inharmonicity estimation from one fret.
% %
% \subsection{String Estimation} % (fold)
% \label{sec:string_estimation}
% Insert overall average performance for both guitars. (in a table)
%%%%%%%%%%%%%%%%%%%%%%%%%%%%%%%%%%%%%%%%%%%%%%%%%%%%%%%%%%%%%%%%%%%%%%%%%%%%%%%%
%
%
% --------- EVALUATION OF CLASSIFIER
%
%
\subsection{Classification of String and Fret} % (fold)
\label{sec:string_fret_classification}
\vspace{-.6mm}
The classification is tested on recordings of both guitars. 
%~\eqref{eq:B_definition} and~\eqref{eq:omega0_fast} are used in a $13$-fold hold out scheme where one model is trained on data one fret and classify the remaining data 12 frets.
The classifier trains a string and fret model using~\eqref{eq:B_definition} and \eqref{eq:omega0_fast} iteratively, such that $13\times12\times6 = 9360$ recordings are classified in total. The resulting confusion matrices in Table.~\ref{tbl:string_confusion_martin} and Table.~\ref{tbl:str_confusion_firebrand} are shown here for the strings, and we observe that the acoustic guitar has a very low error rate while for the electric guitar it is on average $3\%$. The biggest confusion occurs between strings 3 and 4. We have observed that the amplitudes for $m>>1$ of the acoustic guitar has relatively more energy than the electric guitar, which is related to string material, acoustic qualities and the electrical systems~\cite{fletcher:physics_of_musical_instruments}.  
The errors for the electric guitar are explored in Fig.~\ref{fig:err_vs_frets}. The lowest error rate would be $1\%$ from the $12$th fret, while it is $5\%$ from training in the $8$th fret. \vspace{-.6mm}
%
\begin{table}\centering \vspace{-1mm}
\caption{String and fret confusion matrix for the Martin  acoustic guitar. The classification errors are shown for each of the 6 strings.}
\label{tbl:string_confusion_martin}
\begin{tabularx}{0.46\textwidth}{@{}l*{7}{c}c@{}}
\toprule
Labels &Est. 6   &Est. 5 &Est. 4   &Est. 3   &Est. 2   &Est. 1   \\ 
\midrule
True 6   &1560 \cellcolor[gray]{.8} &0  &0  &0  &0  &0  \\  
True 5   &0  & 1560\cellcolor[gray]{.8} & 0   &0  &0  &0  \\
True 4   &0  &0  &1560 \cellcolor[gray]{.8} &0  &0  &0  \\  
True 3   &0  &0  &0  &1560 \cellcolor[gray]{.8} &0  &0  \\  
True 2   &0  &0  &0  &0  &1560 \cellcolor[gray]{.8} &0  \\  
True 1   &0  &0  &0  &0  &0  &1560 \cellcolor[gray]{.8} \\  
\bottomrule
\end{tabularx}
\end{table}\vspace{-.6mm}
%
\begin{table}\centering \vspace{-.6mm}
\caption{String and fret confusion matrix for the Firebrand electric guitar. The classification errors are shown for each of the 6 strings}
\label{tbl:str_confusion_firebrand}
\begin{tabularx}{0.46\textwidth}{@{}l*{7}{c}c@{}}
\toprule
Labels &Est. 6   &Est. 5 &Est. 4   &Est. 3   &Est. 2   &Est. 1   \\ 
\midrule
True 6   &1536 \cellcolor[gray]{.8}       & 24                        &0      &0  &0  &0 \\
True 5   &0  & 1560\cellcolor[gray]{.8}   & 0                        &0      &0  &0  \\
True 4   &0  &0  &1475 \cellcolor[gray]{.8}                           & 85   &0  &0  \\
True 3   &0  &0  &76                     &1455 \cellcolor[gray]{.8}   & 29   &0  \\
True 2   &0  &0  &0  &0                   &1506 \cellcolor[gray]{.8}  & 54  \\
True 1   &0  &0  &0  &0                   &0                          &1560 \cellcolor[gray]{.8} \\
\bottomrule \vspace{-6mm}
\end{tabularx}
\end{table}
%
\begin{figure}[t]
  \centering
  \centerline{\includegraphics[width=.9\columnwidth]{img/tablature_constant_note8}}\vspace{-2mm}
  \caption{Classification and plucking position estimates on 12 seconds recording of the Firebrand guitar.}\label{fig:pluck_position_fixed_tabs}
\vspace{-.6mm}
\end{figure}\vspace{-.6mm}
%
\begin{figure}[t]
  \centering
  \centerline{\includegraphics[width=.92\columnwidth]{img/tablature_constant_note23_LSD}}\vspace{-2mm}
  \caption{Classification and plucking position estimates on 12 seconds recording of the Firebrand guitar.}\label{fig:pluck_position_varied_tabs}\vspace{-.6mm}
\end{figure} \vspace{-.8mm}
%
\vspace{-.6mm}
\begin{figure}[htbp]
  \centering \vspace{-6mm}
  \centerline{\includegraphics[width=.95\columnwidth]{img/errorRate_vs_frets}}\vspace{-2mm}
  \caption{Classification error rate as a function of the fret that is used for training. Every  dot in the graph represents 720 classifications.}\label{fig:err_vs_frets}
\end{figure}%
%
%
% ----- EVALUATION OF PLUCKING ON FRETTED STRING
%
%
\subsection{Plucking Position Estimation} % (fold)
\label{sec:string_estimation}
The estimation of plucking position is tested on two 12 second recordings of electric guitar that resembles realistic playing with both hands continously during each recording. Since the ground truth of plucking position is demanding to obtain, the experimental ground truth is based on a continously moving the plucking event location; monotonuosly moving it from the bridge and towards the nut of the guitar. The result in Fig.~\ref{fig:pluck_position_varied_tabs} represents a case where 6 strings are being played in six frets in a structured manner. All the string and fret combinations was correctly extracted, except from one missing at the given onset events and the plucking position estimates shows a clear trend that follows the ground truth direction. In Fig.~\ref{fig:pluck_position_fixed_tabs} similar results are shown for a fixed string and fret combination, where we can observe a clear trend in plucking position direction and it is interesting that spatial aliasing occurs from 7 seconds and onwards. \vspace{-.6mm}
%
%
%
% ---- CONCLUSION
%
%
%
\section{Conclusion} 
\label{sec:conclusion}
\vspace{-.6mm}
In this paper a method was proposed for the estimating the plucking position along the fretted guitar string; which includes estimation of the activated string and fret. The plucking position estimator is a minimizer of the log spectral distance between the estimated amplitudes of the observed signal and the plucking model and was evaluated on recordings of various combinations of string, fret and plucking position combinations, where a clear trend was demonstrated for correct plucking position and string and fret combinations. The feature set of the classifier is directly based on non linear least squares pitch estimation, extended to include inharmonicity; derived from a physically meaningful model. A maximum a posteriori classifier was used for the string and fret classification, which was tested by iteratively training on audio from of each string and classifying all the remaining recordings. The accuracy of the classifier was evaluated on both electric and acoustic guitar recordings and yields accurate results with an overall average error rate of $1.4\%$ on 18720 string and fret classifications. The classification errors are dependent on the training fret. The method is derived with physically meaningful parameters that reliably extract the parameters of both hands. The classification performance was lowest for electric guitar which might be improved by including the pickup in the model. 
%
%
%
% ---- BIBLIOGRAPHY
%
%
\vfill\pagebreak
\bibliographystyle{IEEEtran}
%
\bibliography{myabbr,refs}
\end{document}
%
%
% ---- SOME MORE EXPERIMENTAL RESULTS
%
%
% \begin{figure}[t]\label{fig:inharmonicity1}
%   \centering
%   \centerline{\includegraphics[width=\columnwidth]{img/spectral_decay_EString_349_long_file_partial_amplitudes}}
%   \caption{Decay of each estimated amplitude.}
% \end{figure}
% \begin{figure}[t]\label{fig:inharmonicity1}
%   \centering
%   \centerline{\includegraphics[width=\columnwidth]{img/spectral_decay_EString_349_long_file_inh_coff}}
%   \caption{Inharmonicity Estimates over time.}
% \end{figure}
%
%
% \begin{figure}[t]\label{fig:err_vs_frets}
%   \centering
%   \centerline{\includegraphics[width=\columnwidth]{img/pluck_est_stats_strings}}
%   \caption{Plucking position estimates normalized by their estimated mean for each string for the Firebrand electric.}
% \end{figure}
% \begin{figure}[t]\label{fig:err_vs_frets}
%   \centering
%   \centerline{\includegraphics[width=\columnwidth]{img/pluck_est_stats_strings_martin}}
%   \caption{Plucking position estimates normalized by their estimated mean for each string for the Martin acoustic guitar.}
% \end{figure}
%\tabularnewline
%
%
%
% \begin{figure}[t]\label{fig:spectral_decay}
%   \centering
%   \centerline{\includegraphics[width=\columnwidth]{img/spectral_decay_EString_349_long_file4}}
%   \caption{Power of the estimated amplitudes, estimated with the proposed method from a short time Fourier transform (STFT) on a plucked string on the Martin acoustic guitar. (showed in dB).}
% \end{figure}\vspace{-.6mm}%